\documentclass[a4wide,12pt]{scrartcl}
\usepackage{fancyhdr}
\usepackage{alltt}
\usepackage[T1]{fontenc}
\usepackage[utf8]{inputenc}
\usepackage[french]{babel}
\usepackage{xspace}
\usepackage{hyperref}

\selectlanguage{french}
\setlength{\parskip}{2ex plus 1ex minus 1ex}
\widowpenalty=10000
\clubpenalty=10000

\newcommand{\nomAssoc}{PromoLily.ORG\xspace}
\newcommand{\dateStatuts}{\today}

\hypersetup{pdftex, colorlinks=true, linkcolor=blue, citecolor=blue, filecolor=blue, urlcolor=blue, pdftitle=Statuts de \nomAssoc au \dateStatuts, pdfauthor=, pdfsubject=\nomAssoc, pdfkeywords=}

\renewcommand{\thesection}{Article~\arabic{section}}
\renewcommand{\thesubsection}{\arabic{section}\alph{subsection}}

\pagestyle{fancyplain}
\lhead{Statuts de l'association \nomAssoc}
\chead{}
\rhead{\dateStatuts}

\begin{document}

\section{Fondation de l'association \nomAssoc}

Les soussignés :
\begin{itemize}
\item M. Bertrand Bordage, demeurant à Quincampoix (76230),
\item M. Jean-Charles Malahieude, demeurant à Sainte-Tulle (04220),
\item M. John Mandereau, demeurant à Bagnolet (93170),
\item M. Nicolas Sceaux, demeurant à Bordeaux (33800),
\item M. Xavier Scheuer, demeurant en Belgique,
\item M. Michael Solomon, de nationalité étasunienne,
  demeurant à Clichy (92110),
\item M. Valentin Villenave, demeurant à Paris (75011),
\end{itemize}
déclarent fonder une association régie par la loi du 1\ier{}~juillet~1901
et le décret du 16~août~1901, selon les présents statuts, et ayant
pour titre \nomAssoc.


\section{Objet de l'association \nomAssoc}

\subsection{Les objectifs}

L'association se donne pour objectif la promotion de l'utilisation et
du développement du logiciel GNU LilyPond, qui pourra être désigné par
raccourci «~LilyPond~» dans la suite du texte. Par suite,
l’association vise à rassembler développeurs, traducteurs et
utilisateurs de GNU LilyPond.

Étant donné que la communauté de développement et la communauté des
utilisateurs se sont formés en grande partie à travers internet, ces
communautés sont naturellement internationales. Ainsi, et bien que
l'association soit fondée en France, son champ d'action peut être
international, dans les limites fixées par le règlement intérieur,
notamment en fonction de l'existence d'une ou plusieurs associations
dans d'autres pays poursuivant les mêmes objectifs que \nomAssoc.

Tout membre de l'association doit adhérer aux objectifs de \nomAssoc.

\subsection{Réalisation des objectifs}

Pour réaliser ses objectifs, sans exclure d'autres activités connexes
et assimilées conformes à son objet et ses buts, l'association pourra~:
\begin{itemize}
\item proposer et mettre en œuvre des moyens matériels et des services
  d'aide au développement (hébergement de site, serveur), sur
  proposition des développeurs~;
\item organiser ou participer à des actions de formation à
  l'utilisation de GNU LilyPond, y compris en proposant des formateurs
  bénévoles ou salariés~;
\item confier totalement ou en partie la réalisation de certains
  objectifs de l'association à d'autres associations, ou coordonner
  cette réalisation avec d'autres associations, lorsque la
  mutualisation des moyens avec d'autres projets, notamment des
  projets de logiciels libres, rend ces objectifs réalisables ou
  apporte une plus grande efficacité.
\end{itemize}

L'association agira en particulier dans le respect de l'organisation
existante de la communauté LilyPond et en particulier des
développeurs. Des modalités plus précises de réalisation des objectifs
par les membres peuvent être définies dans le règlement intérieur.

\section{Siège de l'association}

Le siège de l'association est situé au domicile du président. Il
pourra être transféré par simple décision du conseil d'administration.


\section{Durée}

La durée de l'association est illimitée.\\ 
L'assemblée générale peut délibérer de sa pérennité.

\section{Membres de l'association}

L'association est composée de :
\begin{itemize}
\item membres fondateurs ;
\item membres actifs ;
\item membres bénévoles ;
\item membres d'honneur ;
\item membres associés ;
\item membres adhérents.
\end{itemize}

\subsection{Membres fondateurs}

Les membres fondateurs, tels que définis par l'article premier, et
ayant acquitté leur cotisation annuelle formeront le premier conseil
d'administration. Les membres fondateurs sont également membres
actifs, jusqu'à ce qu'ils perdent la qualité de membre.

\subsection{Membres actifs}

Toute personne physique peut être membre actif de l'association, à
condition d'adhérer aux présents statuts et au règlement intérieur, de
ne pas avoir été salarié de l'association pendant les six mois
précédant l'adhésion et de s'acquitter du montant de la cotisation en
vigueur lors de l'année d'inscription. Ce montant est indiqué dans le
règlement intérieur.

Elle sera alors membre actif à part entière pour l'année en cours
telle que définie dans le règlement intérieur.

Une personne morale membre actif de l'association peut se faire
représenter par une personne physique de son choix faisant parti de la
personne morale, ou donner son pouvoir à un membre de l'association,
pour les assemblées générales.

Seules les personnes physiques, membres actifs de l'association
peuvent être élues au conseil d'administration.

\subsection{Membres bénévoles}

Sont reconnus membres bénévoles toutes les personnes physiques ou
morales ayant réalisé au moins une fois dans l'année une activité de
traduction ou de relecture des documentations, ou accomplissant des
tâches administratives pour le compte de l'association, mais dont
l'emploi du temps ou la motivation empêche de participer de façon
régulière aux activités de l'association. Ils sont désignés par
décision du conseil d'administration, cette décision étant validée par
l'assemblée générale suivante. Ils sont dispensés de cotisation. Les
membres bénévoles peuvent assister aux assemblées, mais ne peuvent pas
participer au vote.

\subsection{Membres d'honneur}

Les membres d'honneur sont des personnes physiques nommées par le
conseil d'administration ou par l'assemblée générale délibérant sur
proposition du conseil d'administration.

Ils sont choisis parmi les membres fondateurs ou les personnes ayant
rendu des services signalés à l'association.

L'un des membres d'honneur peut-être nommé président d'honneur par le
conseil d'administration.

Le président d'honneur pourra participer aux assemblées générales et
aux réunions du conseil d'administration où il ne prendra pas part aux
votes.

Les membres d'honneur pourront participer aux assemblées générales où
ils ne prendront cependant pas part aux votes.

\subsection{Membres associés}

L'association peut également s'associer avec d'autres membres.

Toute personne physique ou morale peut être membre associé de
l'association à la condition de respecter l'esprit et la lettre des
présents statuts et de s'acquitter de la participation éventuellement
prévue, telle que définie au règlement intérieur.

Elle sera membre associé soit pour l'année de son admission telle que
définie au règlement intérieur, soit pour telle ou telle opération
ponctuelle.  En ce cas, une convention particulière sera établie,
prévoyant les conditions de cette collaboration.

En particulier, les éventuels salariés de l'association sont membres
associés de droit.

Un membre associé participera par ses connaissances, ses relations et
ses dons, au développement des activités de l'association. Il pourra
également participer aux assemblées générales où il ne prendra
cependant pas part aux votes.

\subsection{Membres adhérents}

Toute personne physique qui ne souhaite pas s’investir activement dans
l’association peut être membre adhérent de l’association. Les membres
adhérents n’ont pas de droit de vote aux assemblées générales et ne
peuvent pas être élus au conseil d’administration.

\section{Conditions d'admission des membres}

Les conditions d'adhésion sont fixées par le règlement intérieur. Le
bureau peut rejeter toute demande d'adhésion si la majorité des deux
tiers de ses membres le décide. Cette décision sera, si le demandeur
non admis ne s'y oppose pas, portée à la connaissance des membres lors
de l'assemblée générale suivante.

\section{Perte de la qualité de membre}

La qualité de membre se perd par :
\begin{itemize}
\item la démission adressée par écrit ou par courrier électronique
  authentifié conformément au règlement intérieur, au président de
  l'association~;
\item le décès ;
\item le non-paiement de la cotisation, comme prévu à l'article 5,
  selon une procédure spécifiée par le règlement intérieur~;
\item la radiation prononcée par le conseil d'administration, pour
  infraction aux présents statuts, pour non-respect du règlement
  intérieur ou pour motif grave portant préjudice moral ou matériel à
  l'association.
\end{itemize}

L'intéressé est invité à fournir des explications écrites au conseil
avant la décision éventuelle de radiation. En cas d'urgence manifeste,
le bureau pourra procéder à la suspension de l'intéressé en attendant
ses explications écrites.

Une décision d'exclusion devra être ratifiée par l'assemblée générale suivante. 
Les personnes physiques ou morales perdant la qualité de membre de
l'association ne pourront agir au sein de l'association ou en son nom
entre la date de la décision du bureau et la décision de ratification
par l'assemblée générale suivante.

En cas de non ratification par l'assemblée générale suivante, la
radiation prendra fin et l'intéressé sera déclaré à nouveau membre de
plein droit.

\section{Les assemblées générales}

L'instance supérieure de l'association est l'assemblée générale des
membres actifs, qui se réunit en session ordinaire une fois par an, et
aussi souvent que nécessaire en session extraordinaire, sur la
convocation du président de l'association, à la demande de la majorité
des administrateurs de l'association, ou à la demande d'au moins 25~\%
des membres actifs.

Les assemblées générales peuvent être organisées, au choix du conseil
d'administration, soit sous forme de réunion de personnes sur le
territoire métropolitain, soit sous forme de réunion à distance
autorisant les mêmes possibilités de notification, convocation,
représentation, discussion et vote, conformément aux conditions
précisées par le règlement intérieur.

Les convocations sont faites au moins quinze jours à l'avance par
notification individuelle indiquant le jour et le lieu de la réunion
et l'ordre du jour dressé par le conseil d'administration. Cette
notification pourra être effectuée par voie électronique selon des
modalités prévues par le règlement intérieur. Il sera également
précisé s'il s'agit d'une réunion de personnes ou d'une réunion à
distance.

Le rapport moral, le rapport financier et le budget prévisionnel,
ainsi que toute information se rapportant à l'ordre du jour pourront
être communiqués aux membres avant l'assemblée générale selon des
modalités prévues par le règlement intérieur.

Tout membre de l'association peut adresser au conseil
d'administration, jusqu'à sept jours avant la date de la réunion, une
proposition d'inscription d'un sujet à l'ordre du jour. Toute
proposition faisant l'objet d'un vote positif d'un quart au moins des
membres de l'association participants, représentés, ou votant par
correspondance sera ajoutée à l'ordre du jour.

L'assemblée générale délibère alors sur tous les points inscrits à
l'ordre du jour ainsi complété.

Les décisions de l'assemblée générale, ordinaire ou extraordinaire,
sont prises à la majorité des membres à part entière participants,
représentés, ou votant par correspondance. Le règlement intérieur
précise les modalités du vote par correspondance.

En cas de partage, la voix du président est prépondérante. Tout membre
actif peut se faire représenter par un autre membre actif et muni d'un
pouvoir comme précisé dans le règlement intérieur. Un membre actif ne
pourra détenir plus de trois pouvoirs.

Pour que l'assemblée générale soit valablement constituée, le quorum,
prenant en compte les membres actifs présents ou représentés, est fixé
à 30~\% du nombre total des membres actifs. Si l'assemblée générale est
appelée à délibérer sur la pérennité de l'association, le quorum est porté
à 50~\%.

En l'absence de quorum, une nouvelle assemblée générale sera tenue
dans un délai de trente jours maximum, selon une convocation
respectant les modalités définies précédemment pour toute assemblée
générale~; elle pourra alors délibérer sans quorum.

L'assemblée générale reçoit le rapport moral et le rapport financier
de l'année écoulée et statue sur leur approbation.

Les membres actifs peuvent y décider d'élire deux membres de
l'association qui ne font pas partie du bureau et qui ne sont pas
nécessairement des membres actifs, selon des modalités précisées dans
le règlement intérieur.

Elle fixe le montant annuel des cotisations et participations, sur
proposition du conseil d'administration. En cas de besoin, elle peut
déléguer au conseil d'administration le pouvoir de fixer ce montant
ultérieurement. Dans tous les cas, ce montant est inscrit dans le
règlement intérieur.

\section{Le conseil d'administration}

L'association \nomAssoc est administrée par son conseil
d'administration.

Le conseil d'administration comprend au moins six membres et au plus
quinze membres. Le nombre de membres est fixé par le règlement
intérieur.

Tout administrateur peut faire inscrire ce qu'il désire à l'ordre du
jour du conseil.

Le conseil d’administration se réunit à la demande d’au moins un de
ses membres.

\subsection{Élection au conseil d'administration}

Le premier conseil d'administration est composé des membres
fondateurs.

Le mandat d'un membre du conseil d'administration se termine à la fin
de l'élection pour le renouvellement de son mandat. Le nouveau conseil
d'administration se réunit dans le mois suivant son élection pour
élire le nouveau bureau. Dans l'intervalle, en cas de non réélection
au conseil d'administration d'un membre du bureau, l'intérim sera
assuré par un autre membre du bureau, ou par défaut, par un membre du
conseil d'administration.

Les administrateurs sont élus par les membres actifs qui sont des
personnes physiques, pour un mandat de trois ans, reconductible.

Pour les trois premières années, la détermination des membres
fondateurs sortants du conseil est effectuée comme suit :
\begin{itemize}
\item la première année, la partie entière du tiers des membres est
  désignée sortante par le sort~;
\item la seconde année, la partie entière de la moitié des membres
  fondateurs qui n'ont pas été sortants la première année est désignée
  par le sort~;
\item la troisième année, sont sortants les membres initiaux restants qui
n'ont jamais été sortants.
\end{itemize}

Les conditions d'éligibilité et les modalités d'élection sont définies
par le règlement intérieur.

Lorsque le conseil d'administration constate la vacance d'un de ses
membres, le conseil pourvoit à son remplacement provisoire par
cooptation. Le conseil d'administration coopte un membre à la majorité
des deux tiers des présents, lors d'une réunion régulièrement
convoquée, conformément au règlement intérieur. Il est procédé au
remplacement définitif d'un membre vacant lors de la première
assemblée générale suivant le constat de vacance.

Nul ne peut faire partie du conseil d'administration s'il n'est pas
majeur.

\subsection{Attributions du conseil d'administration}

Le conseil d'administration assure l'exécution des décisions prises en
assemblée générale.

Il prépare le budget, rédige le compte-rendu moral qui sera lu en
assemblée générale, fait procéder aux convocations des assemblées
générales ordinaires et extraordinaires, et arrête leur ordre du jour.

Il autorise toute acquisition, aliénation ou location immobilière
ainsi que les contrats de toute sorte à intervenir entre l'association
et des personnes physiques ou des personnes morales, de droit public
ou de droit privé, selon des modalités spécifiées par le règlement
intérieur. En particulier, tout acte dont le montant dépasse une somme
fixée par le règlement intérieur sera l'objet d'un vote par le conseil
d'administration à la majorité simple.

Il assure le respect des statuts et du règlement intérieur et, d'une
façon générale, le bon fonctionnement de l'association.

Il statue sur les demandes d'adhésion et sur les exclusions
éventuelles.

Les décisions du conseil d'administration sont prises à la majorité
simple des présents ou représentés. En cas de partage des voix pour
quelque décision que ce soit, celle du président est prépondérante.

En plus des membres conseil d'administration, des membres de
l'association, ou toute personne étrangère à celle-ci, peuvent
assister à des réunions du conseil d'administration à la demande de
celui-ci, dans la mesure où le conseil estime leur présence nécessaire
de par leurs fonctions. Ils n'ont que voix consultative et ne
participent donc pas aux votes.

Les autres modalités de fonctionnement du conseil d'administration
sont indiqués au règlement intérieur.

\section{Le bureau}

\subsection{Élection du bureau}

Le conseil d'administration élit chaque année un bureau comprenant~:
\begin{itemize}
\item un président et, s'il y a lieu, un ou plusieurs
  vice-présidents~;
\item un trésorier et, s'il y a lieu, un trésorier adjoint~; 
\item un secrétaire et, s'il y a lieu, un secrétaire adjoint.
\end{itemize}

Les membres du bureau sont élus par le conseil d'administration, parmi
ses membres, lors de sa première réunion devant se tenir
obligatoirement dans le mois suivant l'assemblée générale
annuelle. Les membres du bureau sont élus pour un mandat de un an
reconductible.

Le bureau sortant assure ses fonctions jusqu'à l'élection du nouveau
bureau.

En cas de conflit entre le président et le conseil d'administration,
une réunion du conseil peut être provoquée par l'un des membres du
conseil d'administration, et l'élection d'un nouveau président mise à
l'ordre du jour.

Le président du bureau est le président de l'association.

\subsection{Attributions du bureau}

Le bureau assure la gestion et l'administration courante de
l'association.

Il prépare également les comptes rendus des réunions du conseil
d'administration, ainsi que l'ordre du jour des réunions à venir.

Il présente au moins une fois par année civile, lors de l'assemblée
générale, la liste de toutes les actions décidées ou autorisées par
les membres du conseil d'administration.

Les décisions du bureau sont prises à la majorité simple. En cas de
partage des voix pour quelque décision que ce soit, celle du président
est prépondérante.

En cas de départ du bureau pour quelque cause que ce soit, le
remplacement du membre sortant sera effectué au cours du conseil
d'administration suivant. La fin du mandat du remplaçant est la même
que celle du membre remplacé.
 
\subsection{Attributions du président et des membres du bureau}

\subsubsection*{Le président}

Le président dirige l'association, convoque et préside les assemblées
générales.

Le président a signature sur tout document engageant la responsabilité
de l'association. Il peut accorder des délégations partielles de ses
pouvoirs et habiliter dans les formes prévues au règlement intérieur
tout membre du bureau ou personne \textit{ad hoc} à signer les documents
comptables et financiers de l'association. Le règlement intérieur
précise les modalités de ces délégations.

Il dirige et convoque également les réunions du conseil
d'administration et du bureau.

Il représente l'association dans tous les actes de la vie civile, et
est investi des pouvoirs à cet effet. Il conclut tout accord avec des
personnes physiques ou morales sous réserve des autorisations qu'il
doit obtenir du conseil. À ce titre, il passe les contrats au nom de
l'association.\\
Le président a qualité pour présenter toute réclamation
auprès de toute administration, notamment en matière fiscale, et pour
ouvrir tout compte bancaire ou postal. Il agit en justice au nom de
l'association, avec l'autorisation du bureau tant en demande qu'en
défense.
 
En cas d'absence ou de maladie, il est remplacé par un vice-président
selon son choix ou, à défaut, choisi par la majorité du conseil
d'administration. Le vice-président remplaçant dispose alors des
mêmes pouvoirs.

\subsubsection*{Les vice-présidents}

Chaque vice-président représente l'association dans tous les actes de
la vie civile et est investi de tous pouvoirs à cet effet, en cas
d'absence, de maladie ou autre empêchement du président, il peut
convoquer les réunions du conseil d'administration.

Les vice-présidents assistent le président dans l'animation et la
supervision des activités de l'association.

\subsubsection*{Le secrétaire}

Le secrétaire est chargé en particulier de rédiger les procès-verbaux
des réunions du conseil et de tenir le registre prévu par la loi.

En cas d'empêchement, le secrétaire est remplacé par le secrétaire
adjoint, ou un membre du conseil désigné par le président.

\subsubsection*{Le trésorier}

Le trésorier est chargé de tenir ou de faire tenir sous son contrôle
la comptabilité de l'association. Il perçoit les recettes et il
effectue tout paiement sous réserve de l'autorisation du président
dans les cas éventuellement prévus par le règlement intérieur. Il
présente un arrêté des comptes annuels en assemblée générale.

En cas d'empêchement, le trésorier est remplacé par le trésorier
adjoint ou un autre membre du conseil désigné par le président.

\section{Gratuité du mandat}

Les membres du conseil d'administration, de même que les autres
membres de l'association, ne peuvent recevoir aucune rétribution à
raison des fonctions qui leur sont conférées. Les membres du conseil
d'administration pourront toutefois obtenir le remboursement des
dépenses engagées pour les besoins de l'association, sur justification
et après accord du bureau. En ce qui concerne les autres membres, le
remboursement des dépenses engagées ne pourra être envisagé que si le
conseil d'administration a approuvé la dépense, préalablement à
l'engagement de celle-ci.

En cas de besoin, le règlement intérieur fixera les modalités ainsi
que les tarifs et plafonds de remboursement.
 
\section{Ressources et cotisations}

Les ressources de l'association sont constituées des cotisations de
ses membres, des dons manuels, des subventions, des participations des
membres associés, ainsi que des produits éventuels de son activité et
de toute ressource qui ne soit pas contraire aux règles en vigueur.

C'est le conseil d'administration qui gère les finances de
l'association au mieux des intérêts de cette dernière.

Le conseil d'administration fixe annuellement le montant des
cotisations et le fait approuver par l'assemblée générale.

\section{Communication interne}

\subsection{Moyens de communication}

Les outils de communication modernes, tels que le téléphone, le
courrier électronique, l'\textit{Internet Relay Chat} (IRC), la
diffusion de flux audio ou les logiciels de travail de groupe, pourront
être utilisés en lieu et place de ou conjointement avec le courrier
traditionnel ou les rencontres directes, pour simplifier le travail du
bureau et du conseil d'administration, ainsi que pour la communication
entre ces derniers et les membres de l'association.

En particulier, la signature numérique de courrier électronique et de
documents sous forme de fichiers informatiques, au moyen de paire de
clés PGP du signataire signée par un au moins un membre du conseil
d'administration après reconnaissance en face à face avec présentation
réciproque d'une pièce d'identité et de l'empreinte de clé est
reconnue avoir autant de valeur qu'une signature manuscrite sur un
document papier autant que la loi le permet, excepté pour la version
initiale des statuts qui doit être signée en version papier.

Ces moyens pourront en particulier être utilisés pour les réunions du
conseil d'administration et du bureau ainsi que lors des assemblées
générales, dans des conditions précisées par le règlement intérieur.

\subsection{Fichiers informatiques}

Des fichiers informatiques pourront être constitués et gérés par les
membres du bureau, exclusivement dans le but de permettre la gestion
des membres et de poursuivre les objectifs de l'association, et selon
des modalités précisées par le règlement intérieur.  Au cas où la loi
l'impose, certains de ces fichiers seront déclarés à la
Commission nationale de l'informatique et des libertés.

Un membre de l'association a le droit d'obtenir, auprès du bureau ou
par le biais d'un accès limité, sécurisé et authentifié selon des
conditions définies par le règlement intérieur, les noms et
coordonnées d'autres membres de l'association, uniquement pour
utiliser celles-ci à des fins en relation avec l'objet de
l'association, y compris lorsqu'il se porte candidat au conseil
d'administration.

En aucun cas un accès aux noms, coordonnées et autres données des
membres ne faisant pas partie du conseil d'administration ne pourra
être accordé à des non membres de l'association, sauf dans les cas
prévus par la loi.

\section{Participation des membres associés}

Le conseil d'administration propose chaque année les conditions de la
participation des membres associés et les fait approuver par
l'assemblée générale.

\section{Utilisation du logo de l'association}

Les membres actifs ou associés peuvent faire référence à leur
affiliation à l'association, à condition d'en respecter les buts et la
déontologie.

L'utilisation du ou des logos de l'association sur un document papier
est soumise expressément à l'accord du président. Sur un document
hypermédia qui respecte l'esprit et la lettre des statuts de
l'association, elle est subordonnée à l'existence d'un lien hypertexte
du logo vers le site officiel de l'association ou vers un miroir de ce
site agréé par l'association.

\section{Représentation et prestation}

Tout acte ou prestation effectué au bénéfice de tiers au nom de
l'association par l'un de ses membres devra être autorisé par le
président. Si l'acte ou la prestation au nom de l'association est
rétribué, il ne pourra donner lieu à rétribution personnelle, sauf
sous forme de salaire versé à un membre associé de l'association,
l'association étant dans ce cas le seul bénéficiaire autorisé en la
personne de son trésorier.

L'embauche de salariés et toute rémunération de prestation pour le
compte de l'association doivent être autorisés par le président ou
toute personne dûment mandatée par lui.

\section{Statuts}

Seule l'assemblée générale a le pouvoir de faire addition ou
modification aux présents statuts qui seront adoptés par elle.

Cette modification ne pourra intervenir qu'à la majorité des deux
tiers des membres votants.

Il pourra être fait une traduction des statuts dans d'autres langues,
mais seul le texte en français des statuts a valeur légale.

\section{Règlement intérieur}

Le règlement intérieur est destiné à fixer les divers points non
prévus par les statuts et utiles à la réalisation des objectifs de
l'association.

Les modifications, proposées par le conseil d'administration, sont
soumises au vote de l'assemblée générale. Cependant, en cas d'urgence,
elles peuvent être adoptées provisoirement jusqu'à leur ratification
par la prochaine assemblée générale, par un vote positif du conseil
d'administration à la majorité des deux tiers.

Il est en permanence tenu à la disposition des adhérents de
l'association.

Il est établi en respect des présents statuts et a force obligatoire à
l'égard de tous les membres de l'association.

\section{Dissolution}

La dissolution de l'association ne peut être prononcée que par une
assemblée générale extraordinaire, convoquée spécialement à cet effet.

Pour ce faire, une majorité des deux tiers des votants doit être
obtenue. Un ou plusieurs liquidateurs seront alors désignés par
l'assemblée générale, qui disposeront des actifs en faveur d'une ou
plusieurs associations sans but lucratif poursuivant des objectifs
analogues.
\end{document}
